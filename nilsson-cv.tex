%!TEX TS-program = xelatex
%!TEX encoding = UTF-8 Unicode

\documentclass[11pt,letterpaper,final]{moderncv}
\usepackage{fontspec}
\moderncvtheme[darkred]{classic}

% DOCUMENT LAYOUT
\usepackage[scale=0.85]{geometry}
\setlength{\hintscolumnwidth}{2cm}
\AtBeginDocument{\recomputelengths}
% FONTS
\usepackage[utf8]{inputenc}
\usepackage[swedish]{babel}
\usepackage[T1]{fontenc}
\defaultfontfeatures{Mapping=tex-text} % converts LaTeX specials (``quotes'' --- dashes etc.) to unicode

% Remove % to set to different fonts
%\setromanfont [Ligatures={Common},Numbers={OldStyle}]{Adobe Caslon Pro}
%\setmonofont[Scale=0.8]{Monaco} 

% ---- CUSTOM AMPERSAND
\newcommand{\amper}{{\fontspec[Scale=.95]{Adobe Caslon Pro}\selectfont\itshape\&}}

% ---- MARGIN YEARS
%\newcommand{\years}[1]{\marginpar{\scriptsize #1}}


% PDF SETUP
% ---- FILL IN HERE THE DOC TITLE AND AUTHOR 
% Personal Inofrmation
\firstname{David}
\familyname{Nilsson}
\address{Kungshamra 48}{Solna, 17070}
\mobile{0761162066}
\email{nilsson.dd@gmail.com}

\title{David Nilssons CV}

\nopagenumbers{}

\begin{document}

\maketitle
\section{Utbildning}
% \cventry{2009-2014 (f\"{o} rv\"{a} ntad)}{Civilingenj\"{o} r i Datateknik, KTH, Stockholm}
\cventry{2009-2014 (f\"{o}rv\"{a}ntad)}{Civilingenj\"{o}r i Datateknik, KTH}{Stockholm}{}{}{}
\cventry{2006-2009}{Teknisk inriktning, Oscarsgymnasiet}{Oskarshamn}{}{}{}
\section{Kunskaper} 
  \cvline{Spr\aa k:}{C, C++, Assembler, Java, Matlab, Haskell, Verilog}
  \cvline{Arkitekturer:}{X86, AVR, NIOS II, FPGA, GPU (CUDA)}
  \cvline{Verktyg och milj\"{o}er :}{Windows, GNU/Linux, AVRStudio, Altera Quartus}
  \cvline{Inom mjukvara :}{Programmering p\aa\ l\aa gniv\aa , s\"{a}kerhetsanalys av mjukvara, kryptografi, design och implementation av st\"{o}rre system}
  \cvline{Inom h\aa rdvara :}{Embedded utveckling, konstruktion av digital elektronik, fels\"{o}kning, l\"{o}dning}
\section{Arbetslivserfarenhet} 
  \cventry{Februari 2012 -- }{Embedded developer}{Optistring Technologies AB}{Stockholm}{}{
    \begin{itemize}
      \item Design och implementation av mjukvara p\aa \ inbyggda system.
    \end{itemize}
  }
  \cventry{Januari 2012 -- \\Februari 2012}{Konsult inom embedded development}{Egen verksamhet}{Stockholm}{}{
    \begin{itemize}
      \item Design och implementation av l\aa gniv\aa -mjukvara, bibliotek och visualiseringsprogram.
      \item Arbete med RF och inh\"{a}mtning av sensordata.
    \end{itemize}
  }
  \cventry{Sommaren 2011}{L\"{o}neadministrat\"{o}r}{Elajo Invest}{Oskarshamn}{}{
    \begin{itemize}
      \item Rutinarbete inf\"{o}r l\"{o}nek\"{o}rning och registrering av inkommande dokument.
    \end{itemize}
  }
  \cventry{December 2008 --\\Augusti 2009}{L\"{o}neadministrat\"{o}r}{Elajo Elteknik AB}{Oskarshamn}{}{
    \begin{itemize}% \itemsep -2pt % reduce space between items
      \item Arbete i olika l\"{o}nesystem och extra hj\"{a}lp vid l\"{o}nek\"{o}rningar.
    \end{itemize}
  }
  \cventry{Juni 2008 --\\Juli 2008}{Vaktm\"{a}stare}{Elajo Invest AB}{Oskarshamn}{}{
    \begin{itemize}% \itemsep -2pt % reduce space between items
      \item Arbete med r\"{o}jning och allm\"{a}na vaktm\"{a}starsysslor.
    \end{itemize}
  }
\section{Spr\aa k} 
  \cvline{}{Hanterar b\aa de engelska och svenska flytande}
\section{Extra}
  \cvlistitem{Utvecklar open source - \url{https://github.com/davnils}}
\end{document}
